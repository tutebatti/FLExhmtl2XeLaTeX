
\documentclass[10pt, a4paper, twocolumn, twoside]{book}

\usepackage{polyglossia, xunicode}

\usepackage[dvipsnames]{xcolor} % used for coloring, option 'dvipsnames' provides extended color names
\usepackage{soul} % used for highlighting via \hl

\usepackage[hidelinks]{hyperref} % adapt internal and external linking

\usepackage{graphicx}

\usepackage[]{mdframed} % used for sidebars
\definecolor{lightgreen}{RGB}{230, 240, 194}
\newmdenv[linecolor=gray,backgroundcolor=lightgreen]{sidebar}

\setdefaultlanguage{english}
\setmainfont{Charis SIL}

\setotherlanguage{syriac}
\newfontfamily\syriacfont[Script=Syriac, Scale=1.0]{Serto Antioch Bible}
\newfontfamily\estrangelafont[Script=Syriac, Scale=1.0]{Estrangelo Edessa}
\newfontfamily\madnhayafont[Script=Syriac, Scale=1.0]{East Syriac Adiabene}

\setotherlanguage{arabic}
\newfontfamily\arabicfont[Script=Arabic, Scale=1.0]{Amiri}

\setotherlanguage{hebrew}
\newfontfamily\hebrewfont[Script=Hebrew, Scale=1.0]{Ezra SIL}

\setotherlanguage{amharic}
\newfontfamily\amharicfont[Script=Ethiopic, Scale=1.0]{Abyssinica SIL}

\setotherlanguage[variant=ancient]{greek}
\newfontfamily\greekfont[Script=Greek]{SBL Greek}

\usepackage{setspace} % adjust for linespacing (note: Arabic can cause problems because of exceeding lineheight)
\onehalfspacing
\lineskiplimit=-\maxdimen

\setlength{\parindent}{0pt} % deactivate automatic indentation in new lines

\begin{document}
    

    %%%%%%%%%%%%%%
    %%% New entry
    %%%%%%%%%%%%%%

    \hypertarget{g968bc047-871e-4149-b316-0be8c4fabb93}{}

    \bigskip{}\begin{huge}\RLE{\hyperlink{g968bc047-871e-4149-b316-0be8c4fabb93}{\textsyriac{ܐܶܠܳܐ}}\begin{large}\textsubscript{1}\end{large}}\end{huge} [\href{https://sedra.bethmardutho.org/lexeme/get/128}{Sedra-ID 128}] m. Cf. Simtho Etym Related Lexemes: \textsyriac{\hyperlink{ge4f9ab4a-9488-438a-9de9-4cad1028b559}{\RLE{ܝܠܠ}}}

    

    \smallskip
    \textbf{Aramaic etymologies}

    \begin{footnotesize}The verbal root is a more marginal root variant of \textit{yll/ˀll} with the same meaning. \end{footnotesize}

        \textbf{M} vb. \textit{ˀlˀ\textup{,} ˁlˀ} \textbf{I} `to shriek, wail, howl, lament'  (only act. part.) [MD 18]~| \textbf{JPA} \textit{ˀly} \textbf{I} `to wail'  (only act. part.) [DJPA 59]~| 

        \smallskip
        \textbf{Semitic etymologies}

         \textbf{Hbr} vb. \textup{\texthebrew{אָלָה}} \textit{ˀālā} `wail' (hapax legomenon) [BDB 46, KB 49]

    \begin{footnotesize}See also: [DRS 20].\end{footnotesize}

    \medskip{}
    \textbf{Senses}

    

    \textit{v}

    \textbf{East Syriac \textsyriac{\RLE{ܐܰܠܳܐ}}}

    %%%%%%%%%%%
    %%%subentry
    %%%%%%%%%%%

    \medskip{} --- 
    \begin{Large}\textsyriac{\hyperlink{ga3c40da0-c81f-47f1-a7d9-a28ccce12edb}{\RLE{ܐܶܠܳܐ \textenglish{Ia}}}}\end{Large} [\href{https://sedra.bethmardutho.org/lexeme/get/0}{Sedra-ID 0}] I

    \medskip{}
    \textbf{Senses}

    

    \textit{v}

    \textbf{to mourn, lament}

    %%%%%%%%%%%
    %%%subentry
    %%%%%%%%%%%

    \medskip{} --- 
    \begin{Large}\textsyriac{\hyperlink{gfcf84d88-f002-4247-825a-0fa99836875d}{\RLE{ܐܶܬܶܐܠܺܝ}}}\end{Large} [\href{https://sedra.bethmardutho.org/lexeme/get/0}{Sedra-ID 0}] It

    \medskip{}
    \textbf{Senses}

    

    \textit{v}

    \textcolor{Dandelion}{Y. Kirilenko (Sem. etymology)}

    \textcolor{Dandelion}{A. Cherkashina (Aram. etymology)}
\end{document}
    